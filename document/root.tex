\documentclass[11pt,a4paper]{article}
\usepackage[T1]{fontenc}
\usepackage{isabelle,isabellesym}

% this should be the last package used
\usepackage{pdfsetup}

% urls in roman style, theory text in math-similar italics
\urlstyle{rm}
\isabellestyle{it}


\begin{document}

\title{Abstract Inference Triples}
\author{Matt Griffin \and Brijesh Dongol \and Azalea Raad}
\maketitle

\begin{abstract}
  This entry formalises (in)correctness triples over abstract big-step execution semantics. 
  The aim is to provide a common base for triple-style (in)correctness logics, reducing syntax and 
  theory duplication.

  This work was developed as part of the POPL 2024 paper ``IsaBIL: A Framework for Verifying 
  (In)correctness of Binaries in Isabelle/HOL'' by Matt~Griffin, Brijesh~Dongol and 
  Azalea~Raad~\cite{LNCS2283}.

  This entry defines three triples: \emph{partial correctness}, total \emph{correctness} and \emph{incorrectness}.
  Formalised in locales, these triples encode common inference rules, such as pre/postcondition 
  strengthening/weakening and conjunction/disjunction, which apply to all instantiations of the 
  triples regardless of the execution semantics.

  In order to verify the accuracy of the triples, and to motivate their use, we this entry presents 
  various instantiations of the triples for the numerous (in)correctness logics found
  in the Isabelle/HOL standard library and the AFP.

  Our hope is that in future, new logics for (in)correctness will use these abstract triples.

\end{abstract}

\tableofcontents

% include generated text of all theories
\input{session}

\section{Examples}

We validate our XYZ...

\bibliographystyle{abbrv}
\bibliography{root}

\end{document}
