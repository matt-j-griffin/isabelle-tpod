\documentclass[11pt,a4paper]{article}
\usepackage[T1]{fontenc}
\usepackage{isabelle,isabellesym}

% this should be the last package used
\usepackage{pdfsetup}

% urls in roman style, theory text in math-similar italics
\urlstyle{rm}
\isabellestyle{it}


\begin{document}

\title{TPOD}
\author{Matt Griffin \and Brijesh Dongol \and Andrei Popescu}
\maketitle

\begin{abstract}
  This entry formalises Cheang et al.'s Trace Property-Dependent Observational Determinism (TPOD)
  hyperproperty for categorising secure speculation in Isabelle/HOL. 

  This work was developed as part of 2024 thesis ``Binary Analysis in Isabelle/HOL'' by 
  Matt~Griffin~\cite{LNCS2283}.

  The aim of this development is to allow for high-assurance, unwinding style proofs of TPOD in an
  abstract setting.

  Following Cheang et al.'s approach, we start by defining Observational Determinism (OD), a well
  known two-trace hyperproperty from the literature which underpins TPOD.
  In a departure from Cheang et al., we define a variant of TPOD which reasons over 
  two systems (as opposed to one) which provides greater proof flexibility.
  We extend TPOD with an antecedent trace property $U$ used to conform traces as described by 
  Cheang et al. as Conformant TPOD (CTPOD).
  For validity, we prove that our ``two-system'' TPOD hyperproperties are equivalent to 
  Cheang et al.'s definitions under the assumption that the two systems are equal.


  TODO: statewise - we use a common system model with finality, formalised in Relative secuirty

  To derive unwinding proof methods for OD and TPOD, we leverage Popescu et al.'s BD Security \cite{BD}.
  Existing definitions are not enough, so we define a new BD security property for all for all
security and give it an unwinding proof method.
We now prove that it is an instance of Popescu et al.'s BD security.


  Our definitions of TPOD are abstract, levereaging locales in its development.
  It is our hope that others will instantiate these locales with their own language semantics to 
  prove secure speculation.

\end{abstract}

\tableofcontents

% include generated text of all theories
\input{session}

\bibliographystyle{abbrv}
\bibliography{root}

\end{document}
